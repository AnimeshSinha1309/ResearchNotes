\chapter{Group Theory and Graphs}



\section{Groups and Subgroups}

\subsection{What is a Group?}
A group is defined over a Set A and an arbitrary operation $\times$, denoted as: $\braket{A|\times}$.
\begin{itemize}
    \item \textbf{Closure:} If a and b are elements in A, then $a \times b$ is also in A.
    \item \textbf{Identity:} There exists e such that $a \times e = a$.
    \item \textbf{Invertability:} There exists $a^{-1}$ such that $a \times a^{-1} = e$.
    \item \textbf{Associativity:} $(a \times b) \times c = a \times (b \times c)$
\end{itemize}

A Subgroup is a subset of the original group that is itself a group.
\paragraph{One Step Subgroup Test} states that if $ab^{-1}$ is in the group H, then H is a subgroup of G.
\paragraph{Two Step Subgroup Test} states that if $a^{-1}$ is in H whenever a is in H and $ab$ is in H for all a, b in H, then H is a subgroup of G.
\paragraph{Finite Subgroup Test} If H is a non-empty finite subset of a group G, and H is closed under the operation G, then H is a subgroup of G.

\paragraph{Operations that Hold} in groups are:
\begin{itemize}
    \item Uniqueness of Identity (If $ x \cdot a = x $ and $ x \cdot b = x $, $(\forall x)$, then $a = b = e$)
    \item Uniqueness of Inverse (If $ x \cdot a = e $ and $ x \cdot b = e $, $(\exists x)$, then $a = b = x^{-1}$)
    \item Left and Right Cancellation (If $ab = ac$ then $b = c $. If $ba = ca$ then $b = c$.)
    \item Socks-Shoes Property ($(ab)^{-1} = b^{-1} a^{-1}$)
\end{itemize}


\subsection{Cayley's Table}

\paragraph{Cayley's Table} is a 2-D matrix of all members of the group $a$ and $b$ on both axis and $a \cdot b$

\subsection{Subgroups and GCD}

\begin{theorem}[Equivalent Cyclic Subpgroups]{thm:GroupTheory-CyclicGCD}
    Let a be an element of order n in a group and let k be a positive integer. Then 
    \colorbox{yellow}{$ \braket{a^k} = \braket{a^{gcd(n,k)}}\;and\;|a^k| = n/gcd(n, k). $}    
\end{theorem}

\begin{proof}[Equivalent Cyclic Subpgroups]{prf:GroupTheory-CyclicGCD}
    $(a^{gcd(n,k)})^{\alpha} = a^k$, Since gcd(n,k) divides k. Thereby $ \braket{a^{k}} \subseteq \braket{a^{gcd(n,k)}}$.
    Also, $gcd(n,k) = \alpha n + \beta k$, so $a^{gcd(n,k)} = a^{\alpha n + \beta k} = a^{\alpha n} a^{\beta k} = e \cdot a^{\beta k}$, therefore we can state that, $\braket{a^{gcd{n,k}}} \subseteq \braket{a^k}$. So we proved that $\braket{a^k} = \braket{a^{gcd(n,k)}}$.
    \vspace{10pt} \\ Next, using the proof in the first part, since the groups are equal their orders are the same, so
    $|a^{k}| = |a^{gcd(n,k)}| = \frac{n}{gcd(n,k)}$, since the gcd divides n, it is the least solution to $(a^{gcd(n,k)})^x = a^n$
\end{proof}

This has the following crucial corollaries.
\begin{itemize}
    \item In a finite cyclic group, the order of an element divides the orderof the group.
    \item Let $|a| = n$. Then $\braket{a^i} = \braket{a^j}$ if and only if $gcd(n, i) = gcd(n, j)$, and $|a^i| = |a^j|$ if and only if $gcd(n, i) = gcd(n, j)$.
    \item Let $|a| = n$. Then \colorbox{yellow}{$\braket{a} = \braket{a^j}$ if and only if $gcd(n, j) = 1$}, and $|a| = |\braket{a^j}|$ if and only if $gcd(n, j) = 1$.
    \item An integer k in $Z_n$ is a generator of $Z_n$ if and only if $gcd(n, k) = 1$.
\end{itemize}
These facts help us count the number of subgroups in a given set.


\subsection{Cyclic Groups}

\begin{theorem}[Fundamental Theorem of Cyclic Groups]{thm:GroupTheory-FundamentalTheoremCyclicGroups}
    Every subgroup of a cyclic group is cyclic.
    Moreover, if $|\braket{a}|$ = n, then the order of any subgroup of $\braket{a}$ is a divisor of n; and, for each positive divisor k of n, the group $\braket{a}$ has exactly one subgroup of order k, namely, $\braket{a^{n/k}}$.
\end{theorem}

\begin{theorem}[Number of Elements of Order D]{thm:GroupTheory-OrderD}
    Let $G_n$ be a finite Cyclic Group of order N, if d is a positive Divisor of N, then the number of elements of order D in G of order d is $\phi(d)$. \vspace{5pt} \\ For any group (i.e. Including non-cyclic), the number of elements of order d is a multiple of $\phi(d)$.
\end{theorem}

\begin{proof}[Number of Elements of Order D]{prf:GroupTheory-OrderD}
    If d is a divisor of n, then there is only one subgroup of order D of the group $G_n$. The elements in that group are those which are of the form $ \braket{a^x} $, such that D and x are coprime. Therefore there can only be exactly $\phi(d)$ elements that are of order d.
    \vspace{5px} \\ If there are no elements of order D in the group, $\phi(d) | 0$. Now let $\braket{a_1}$ be a subgroup of order d, it has $\phi(d)$ elements in the subgroup of order d. So on for each $\braket{a_i}$ for all i, therefore a multiple of $\phi(d)$.
\end{proof}


\subsection{Abelian Groups}

\begin{theorem}[Fundamental Theorem of Abelian Groups]{thm:GroupTheory-AbelianGroups}
    Every finite Abelian group is a direct product of cyclic groups of prime order power.
    Moreover, the number of terms in the product and the orders fo the cyclic groups are 
    uniquely determined by the group. This is valid \textbf{upto isomorphism}.
\end{theorem}

\begin{proof}[Fundamental Theorem of Abelian Groups]{prf:GroupTheory-AbelianGroups}
    \paragraph{Lemma 1} Let G be a finite Abelian Group of order $p^n m$, where p is a prime that does not divide m. Then $ G = H \times K $, where $ H = \{ x \in G | x^{p^n} = e \} $ and $ K = \{ x \in G | x^{m} = e \} $. Moreover $ |H| = p^n $
    \paragraph{Lemma 2} Let G be an Abelian group of prime-power order and let a be an element of maximum order in G. Then G can be written in the form $\braket{a} = K$
    \paragraph{Lemma 3} A finite Abelian group of prime-power order is an internal direct product of cyclic groups.
    \paragraph{Lemma 4} G be a finite Abelian group of prime-power order. If $ G = H_1 \times H_2 \times ... \times H_m$ and $G = K_1 \times K_2 \times ... \times K_n $, where the H’s and K’s are nontrivial cyclic subgroups with $|H_1| \geq |H_2| \geq ... \geq |H_m| $ and $ |K_1| \geq |K_2| \geq ... \geq |K_n| $, then $m = n$ and $|H_i| = |K_i| \; (\forall i)$.
\end{proof}

\begin{example}[Using Cardinality and Generators]{exm:GroupTheory-AbelianGroups}
    \textbf{Prove that every abelian group of order that is a product of primes is also cyclic.}
    Given any abelian Group of order $ p \cdot q $. If there is no prime such that 
\end{example}



\section{Special Groups and Their Properties}

\subsection{Permutation Groups}

\paragraph{A Dihedral Group ($D_n$)} is a group of all permutations of a n-sided Regular
Polygon. The number of elements in this group is $2^n$.

\begin{definition}[Permutation Groups]{def:GroupTheory-PermutationGroups}
Permutation Groups are a group of permutations, where a permutation is a
bijective function from a group to itself. Eg.
\begin{equation}
    Example Permutation = 
    \begin{bmatrix}
        A_0 & A_1 & A_2 & A_3 & A_4 & ... \\
        \alpha(A_0) & \alpha(A_1) & \alpha(A_2) & \alpha(A_3) & \alpha(A_4) & ... 
    \end{bmatrix}
\end{equation}
\end{definition}

\paragraph{Disjoint Cycle Notation} Any permutation can be written as a product of disjoint cycles.
Each cyclic subgroup is expressed as a separate disjoint cycle.
\begin{equation}
    (1,3)(2,7)(4,5,6)(8) =
    \begin{bmatrix} 
        1 & 2 & 3 & 4 & 5 & 6 & 7 & 8 \\ 
        3 & 7 & 1 & 5 & 6 & 4 & 2 & 8 
    \end{bmatrix}
\end{equation}

These Permutation Cycles can be multiplied together (that is operated by the Group Operator, Function Composition). eg.
\begin{eqnarray*}
    P_1 \times P_2 
    & = & (1,3)(2,7)(4,5,6)(8) \times (1,2,3,7)(6,4,8)(5) \\
    & = & (1,3)(2,7)(4,5,6)(8)(1,2,3,7)(6,4,8)(5) \\
    & = & (1,7,3,2)(4,8)(5,6)
\end{eqnarray*}
We go from Right to Left when multiplying, as function composition applies from Right to left.
We see that $1 \rightarrow 2 \rightarrow 7$, $7 \rightarrow 1 \rightarrow 3$, $3 \rightarrow 7 \rightarrow 2$, $2 \rightarrow 3 \rightarrow 1$ so we get the cycle (1,3,7,2). We continue in this fashion to multiply.
By Convention, we can choose not to write down single element cycles.

\paragraph{A Symetric Group $(S_n)$} is a group of all the permutations over the operator Function Composition. There are n! elements in $S_n$
\paragraph{Every Cycle can be written as a product of Disjoint Cycles} Each disjoint cycle is a permutation, and it can always be written as a composition of two or more permutations. Also, any \textbf{disjoint cycles can always commute}.
\vspace{20px}

\paragraph{Order of a permutation} is defined as the Least Common Multiple of the lengths of the disjoint cycles.
\paragraph{Every Cycle can be Written as a Product of 2 cycles} Any permutation is a composition of flips. Also by direct computation, we can prove this. Here is an example: (01234)(56)(789) = (43)(42)(41)(40)(56)(97)(98)
\\ The representation of any cycle as a composition of 2 cycles can be classified as even or odd, i.e. any cycle C will require either even number of 2-cycles in all possible breakdowns or odd 2-cycles in all possible breakdowns, but it can never be that some 2-cycle decompositions are odd order and some are even order.


\subsection{Cosets and Lagrange's Theorem}

\paragraph{A Coset of a subgroup H in group G} is the set $\{ah\;|\;H \in a\}$ - Left Coset, or $\{ha\;|\;H \in a\}$ - Right Coset.

\paragraph{Properties of Cosets} of subgroup H in group G are:
\begin{enumerate}
    \item $\mathbf{a \in aH}$, since $e \in H$, so $ae \in aH$.
    \item \colorbox{green}{$\mathbf{aH = H}$ \textbf{iff} $\mathbf{a \in H}$}, If $ah = H$ then $a = ae \in aH = H$, Also if $a \in H$ then $aH \subset H$ due to closure, $H \subset aH$ as $h = eh = aa^{-1}h = a(a^{-1}h) \in aH$, as $a^{-1}$ is in H (invertability of a in H and then closure).
    \item $\mathbf{(ab)H = a(bH)}$ \textbf{and} $\mathbf{H(ab) = (Ha)b}$, as Associativity holds.
    \item \colorbox{green}{$\mathbf{aH = bH}$ \textbf{iff} $\mathbf{a \in bH}$}, as if $aH = bH$ then $a = ae \in aH = bH$. The other way, if $a \in bH$ then $aH = (bh)H = b(hH) = bH$ so they are equal sets.
    \item \colorbox{yellow}{$\mathbf{aH = bH}$ \textbf{or} $\mathbf{aH \cap bH = \Phi}$}, Using Property 4, if there exists c in the intersection, then $c \in aH$ and $c \in bH$, so $aH = cH = bH$ thereby $aH = bH$.
    \item $\mathbf{aH = bH}$ \textbf{iff} $\mathbf{a^{-1}b \in H}$, If there is $c \in aH,bH$ so we can say that $a^{-1}c \in H$ and $b^{-1}c \in H \iff c^{-1}b \in H$ due to invertability. Multiplying both elements in the we get $a^{-1}c \cdot c^{-1}b = a^{-1}b \in H$, again by Closure.
    \item \colorbox{yellow}{$\mathbf{|aH| = |bH|}$}, We can show a one-one map $aH \rightarrow bH$ using the cancellation property, i.e. $ah = bh \; (\forall h \in H)$.
    \item $\mathbf{aH = Ha}$ \textbf{iff} $\mathbf{H = aHa^{-1}}$, right-multiply both sides by $a^{-1}$, so $aHa^{-1} = Haa^{-1} = H$.
    \item \textbf{aH is a subgroup of G, iff} $\mathbf{a \in H}$, aH must have identity e to be a subgroup, since $aH \cap eH \neq \phi \implies aH = eH = H$. By Property 2, $a \in H$. Conversely if $a \in H$ then $aH = H$ is a subgroup.
\end{enumerate}

\begin{theorem}[Lagrange's Theorem]{thm:GroupTheory-LagrangeTheorem}
    If G is a finite group and H is a subgroup of G, then $|H|$ divides $|G|$. Moreover, the number of distinct (loft/right) cosets of H in G are $|G|/|H|$.
\end{theorem}
\begin{proof}[Lagrange's Theorem]{prf:GroupTheory-LagrangeTheorem}
    Using the Properties proven above, $|aH| = |bH| = |eH| = |H|$ and $aH = bH$ or $aH \cap bH = \phi$, therefore $|G|$ is divisible by $|H|$.
\end{proof}

\paragraph{} The fruitfulness of cosets, when applied to Permutation Groups is displayed here:
\paragraph{Stabilizer of a Point:} Let G be a group of permutations of Set S. For each i in S, let $stab_G(i) = \{\phi \in G \;|\; \phi(i) = i\}$. We call $stab_G(i)$ the stabilizer of i in G.
\paragraph{Orbit of a Point:} Let G be a group of permutations of a set S. For each s in S, let $orb_G(s) = \{\phi(s) \;|\; \phi \in G\}$. The set $orb_G(s)$ is a subset of S called the orbit of s under G.
\paragraph{Eg.:} G = \{(1),(132)(465)(78),(132)(465),(123)(456),(123)(456)(78),(78)\}
\begin{multicols}{2}[]
    \textbf{Orbits of G}\\
    $orb_G(1)$ = \{1,3,2\}\\
    $orb_G(2)$ = \{2,1,3\}\\
    $orb_G(4)$ = \{4,6,5\}\\
    $orb_G(7)$ = \{7,8\}\\
    \textbf{Stabilizers of G}\\
    $stab_G(1)$ = \{(1),(78)\}\\
    $stab_G(2)$ = \{(1),(78)\}\\
    $stab_G(4)$ = \{(1),(78)\}\\
    $stab_G(7)$ = \{(1),(132)(465),(123)(456)\}
\end{multicols}
\paragraph{Orbit-Stabilizer Theorem} states that $|orb_G(i)| \cdot |stab_G(i)| = |G| \; (\forall i)$.


\subsection{Normal and Factor Subgroups}

\paragraph{Normal Subgroups} are subgroups H of group G, if $aH = Ha \;(\forall A \in G)$, written as $H \lhd G$. i.e., $ah = h^\prime a$, the commutations are fudged a bit, when commuting we are allowed to use a different elements from H.
\\ \textbf{Normal Subgroup Test:} A subgroup H of G is normal in G if and only if $xHx^{-1} \subseteq H \;(\forall x \in G)$.
\paragraph{Factor Groups} (or Quotient Groups) are the groups, such that H is a normal Subgroup of G, $G/H = \braket{\{aH\;|\;a \in G\}\;||\;(aH)(bH) = abH}$. This is because the normal subgroups are such that their left or right cosets are themselves subgroups. The operation is the composition of the operation that generated the cosets.
\vspace{5px} \paragraph{Internal Direct Product:} We say that G is the internal direct product of H and K and write $G = H \times K$ if H and K are normal subgroups of G, $G = HK$, $H \cap K = \{e\}$



\section{Isomorphism and Homomorphism}

\subsection{Isomorphisms}

\begin{definition}[Isomorphism]{}
A Isomorphism is a bijective mapping from one group onto another wherein if $ a \times b = c $ then $ \Phi(a) \times \Phi(b) = \Phi(c) $. That is the mapping preserve the result of every operation, and every inverse.
\end{definition}

\paragraph{Automorphism} is an Isomorphism that exists from a group onto itself. \\
\vspace{10px}
\begin{example}[Disproving Isomorphisms]{exm:GroupTheory-DisprovingIsomorphisms}
    \textbf{The Set of Real numbers can never be isomorphic to proper subset of itself under the operation addition.}
    \paragraph{} The homomorphisms of real numbers to its proper subsets under addition are highly constrained: $ \phi(a) + \phi(a) = \phi(2a) \implies \phi(m) = qm $. Therefore there can only be a multiplicative map that holds the homomorphism.
    \paragraph{} Any subgroup of rational numbers can be generated by a subset of the series if prime inverses: $\{\frac{1}{2},\frac{1}{3},\frac{1}{5},\frac{1}{7},\frac{1}{11},\frac{1}{13},...\}$, the set of all prime denominators. Since both of our subgroups are generated by a subset of this set, there must exist a map, just a multiplicative factor, $\phi$, from $A \rightarrow B$.
    This never exists, since there is no such factor that maps the set of prime numbers to another set of prime numbers (or the reciprocals thereof).
\end{example}
\vspace{10px}
\begin{example}[Proving Isomorphisms]{exm:GroupTheory-ProvingIsomorphisms}
    \textbf{From $\braket{\mathbb{R} | +}$ to $\braket{\mathbb{R} | \times}$, show an isomorphism exists}.
    \paragraph{} We have $\phi(x) = 2^x$, the mapping is bijective, and $\phi(x) \times \phi(y) = 2^x \times 2^y = 2^{x+y} = \phi(x+y)$
\end{example}

\paragraph{Cayley's Theorem} states that every group is isomorphic to a group of permutations.


\subsection{Homomorphism}

\begin{definition}[Group Homomorphism]
    A Homomorphism $\phi$ is a mapping from a group G to $\bar{G}$ is a mapping that preserves the group operation, i.e. $ \phi(ab) = \phi(a)\phi(b) \; (\forall\;a,b \in G)$. \\ 
    (The mapping may not be bijective, as opposed to Isomorphism)
\end{definition}

\begin{definition}[Kernel of a Homomorphism]
    The kernel of a homomorphism $\phi$ from a group G to a group with identity e is the set $\{x \in G\;|\;\phi(x) = e\}$. The kernel of $\phi$ is denoted by $Ker \phi$. \\
    (All elements in G that map to the identity, for isomorphism it's just the identity element - trivial subgroup)
\end{definition}


\section{Rings and Fields}

\subsection{Rings}

\paragraph{A Ring is a set with 2 binary operations on it} 
such that the following properties hold true. (Z is a arbitrary set, and 
$+$ and $\times$ are arbitrary operations in $\braket{ Z | +, \times }$, 
also $+$ is the first and $\times$ is second operation on the ring).
\begin{itemize}
    \item Associativity over +: (a + b) + c = a + (b + c)
    \item Commutativity over +: a + b = b + a
    \item Identity over +: a + 0 = a must exist for some 0
    \item Invertable over +: a + (-a) = 0 must exist for some -a
    \item Associativity over $\times$: $a \times (b \times c) = (a \times b) \times c$.
    \item Distributivity of $\times$ over $+$: $a \times (b + c) = (b + c) \times a = (a \times b) + (a \times c)$.
\end{itemize}
In summary, A Ring is a Abelian Group over +, and Associative and Distributive over $\times$.

\paragraph{A subring S of a ring R} is the same operations defined over a subset of elements such that they in themselves form a ring.
\paragraph{Subring Test} A non-empty subset of a ring is a subring if it is closed unber subtraction ($a - b$) and multiplication ($a \times b$).


\subsection{The Integer Domain}

\paragraph{Rings were invented to abstract the algebraic properties} of Integers. However they lose essential features in this abstraction, those of \textbf{Existance of Unity, Commutativity, and Cancellation}
\paragraph{Zero Divisors} are elements of a Commutative Ring such that there is a non-zero element $b \in R$ with $ab = 0$.
\paragraph{Integral Domain} is a commutative ring with unity and no zero-divisors. (It may also be defined as a Commutative Ring with Cancellation, Equivalent)


\subsection{Fields}

\paragraph{A Field} is a commutative ring with unity in which every nonzero element is a unit (i.e. Invertable).
\\ Every finite integral domain is a field. (Finite Order: So every element is invertible)

\paragraph{Characteristic of a Ring} is the least positive integer n such that nx = 0 for all x in the Ring. If no such integer exists then it is 0.
\\ If R is a ring with unity (1), then the characteristic of n is the order of 1, unless its $\infty$ in which case characteristic is 0.
\\ Characteristic of an integral domain can only be 0 or prime. Its because if the field is finite, then $0 = n \cdot 1 = (p \cdot 1)(q \cdot 1)$. So, either $p \cdot 1 = 0$, or $q \cdot 1 = 0$. Since p or q are smaller than n, then p or q is the characteristic, not n. 
