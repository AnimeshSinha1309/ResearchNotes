\chapter{Classical Mechanics}



\section{Vector Calculus}


\subsection{Vector Algebra}

\paragraph{Scalar Triple Product} between any three vectors is defined as 
\begin{equation*} \vec{A} \cdot \vec{B} \times \vec{C} = \begin{vmatrix}A_x & A_y & A_z \\ B_x & B_y & B_z \\ C_x & C_y & C_z\end{vmatrix} \end{equation*}
\paragraph{Vector Triple Product} between 3 vectors can be simplified down to
\begin{equation*} \vec{A} \times \vec{B} \times \vec{C} = B(A \cdot C)-C(A \cdot B) \end{equation*}

\subsection{Gradient, Divergence, and Curl}

\paragraph{The Gradient} of a scalar field in Cartesian Coordinates is defined as:
\begin{equation}
    \vec{\nabla} \psi = \frac{\partial \psi}{\partial x} \hat{x} + \frac{\partial \psi}{\partial y} \hat{y} + \frac{\partial \psi}{\partial z} \hat{z}
\end{equation}

\paragraph{The Divergence} of a vector field in Cartesian Coordinates is defined as:
\begin{equation}
    \vec{\nabla} \cdot \vec{A} = \frac{\partial A_{x}}{\partial x} + \frac{\partial A_{y}}{\partial y} + \frac{\partial A_{z}}{\partial z}
\end{equation}

\paragraph{The Curl} of a vector field in Cartesian Coordinates is defined as:
\begin{equation}
    \vec{\nabla} \times \vec{A} = 
    \begin{vmatrix}
        \hat{x} & \hat{y} & \hat{z} \\
        \frac{\partial}{\partial x} & \frac{\partial}{\partial y} & \frac{\partial}{\partial z} \\
        A_x & A_y & A_z
    \end{vmatrix} 
\end{equation}


\subsection{Understanding the Fields}

Divergence is the net flux coming out of an infinitesimally small volume per unit area, i.e. .

Curl is the amount of the circulation of a vector field, that is the integral of the vector at each point along the boundary.


\subsection{Differential Operators in Curvilinear Coordinates}

\paragraph{The Gradient} in generalized Curvilinear Coordinates is expressed as:
\begin{equation}
    \vec{\nabla} \psi = \frac{1}{h_1} \frac{\partial \psi}{\partial q_1} \hat{q_1} + \frac{1}{h_2} \frac{\partial \psi}{\partial q_2} \hat{q_2} + \frac{1}{h_3} \frac{\partial \psi}{\partial q_3} \hat{q_3}
\end{equation}

\paragraph{The Divergence} in generalized Curvilinear Coordinates is expressed as:
\begin{equation}
    \vec{\nabla} \cdot \vec{A} = \frac{1}{h_1 h_2 h_3} (\frac{\partial A_{1} h_2 h_3}{\partial q_1} + \frac{\partial A_{2} h_3 h_1}{\partial q_2} + \frac{\partial A_{3} h_2 h_1}{\partial q_3})
\end{equation}

\paragraph{The Curl} in generalized Curvilinear Coordinates is expressed as:
\begin{equation}
    \vec{\nabla} \times \vec{A} = \frac{1}{h_1 h_2 h_3} 
    \begin{vmatrix}
        \hat{q_1} h_1 & \hat{q_2} h_2 & \hat{q_3} h_3 \\
        \frac{\partial}{\partial q_1} & \frac{\partial}{\partial q_2} & \frac{\partial}{\partial q_3} \\
        h_1 A_1 & h_2 A_2 & h_3 A_3
    \end{vmatrix} 
\end{equation}


\subsection{Fundamental Theorems}

\paragraph{The Fundamental Theorem of Gradient} states that the gradient of a scalar field is analytic, so the integral along any path in the field is the same, and equal to difference of value of the scalar field between those two points.
\begin{equation} \int_a^b (\vec{\nabla} T) \cdot d \vec{l} = T(b) - T(a) \end{equation}

\paragraph{The Fundamental Theorem of Divergence} states the the integral of the Divergence over a volume is the same as the closed integral of the vector field over the surface that bounds the said volume. This is the \colorbox{yellow}{\textbf{Gauss Divergence Theorem}}.
\begin{equation} \iiint (\vec{\nabla} \cdot \vec{V}) d\tau = \oiint_{\tau} \vec{V} \cdot d \vec{s} \end{equation}

\paragraph{The Fundamental Theorem of Curl} states that the integral of the Curl over a surface is the same as the closed integral of the vector field over the boundary of said surface.  This is also called the \colorbox{yellow}{\textbf{Stokes Theorem}}
\begin{equation} \iint (\vec{\nabla} \times \vec{V}) \cdot d\vec{s} = \oint_{l} \vec{V} \cdot d \vec{l} \end{equation}



\section{From Newton to Lagrange}


\subsection{Center of mass frame}

\paragraph{}
We are expressing values in the stationary coordinate frame (of L, T) in terms of the the Center of mass frame ($r^\prime$, $v^\prime$, $p^\prime$), and those of the system / center of mass itself (R, V, P).

\paragraph{Angular Momentum} of a group of particles is:
\begin{equation}
    L = \vec{R} \times M \vec{V} + \Sigma_i (\vec{r_i^\prime} \times \vec{p_i^\prime})
\end{equation}
\paragraph{Kinetic Energy} of a group of particles is:
\begin{equation}
    T = \frac{1}{2}Mv^2 + \frac{1}{2} \Sigma m_{i}v_{i}^{\prime 2}
\end{equation}


\subsection{Constraints}

\paragraph{Based on Strictness} Constraints can be:
\begin{itemize}
    \item \textbf{Holonomic Constraints}: Contriants of the form $ f(r_1, r_2, r_3 ... r_n) = 0 $ are called Holonomic. Examples: Motion on a fixed path, rigid body contriants ($(r_i - r_j)^2 - c_ij^2 = 0$).
    \item \textbf{Non-Holonomic Contriants}: Constraints which can be written in the form $ f(r_1, r_2, r_3 ... r_n) = 0 $. Examples: Falling off the surface of a sphere ($ r^2 - c^2 \geq 0 $).
\end{itemize}
Given 3N independent coordinates, and K constraints, we can eliminate the dependent coordinates to get N-K remaining generalized coordinates 

\paragraph{Dependence on Time} for classification of contriants
\begin{itemize}
    \item Rheonomous contriants: Explicitly dependent on time.
    \item Scleronomous contriants: Has no explicit dependence on time.
\end{itemize}



\section{Central Force Problem}


\subsection{Solving Periodic Motion in 1-Dimension}

\begin{definition}{Eleptic Integrals}{}
    Eleptic integrals arose from the problem of solving for the arc length of an ellipse.
    Today they are defined as integrals of the form $ \int^x_c R(t, \sqrt{P(t)}) dt $, 
    \textit{where R is a rational function, P(t) is a polynomial of degree 3 or 4 with no 
    repeated roots, and c is a constant of integration}

    \paragraph{Eleptic Integral of the First Kind} are:
    \begin{equation}
        F(\phi; k) = \int^{\phi}_0 \frac{d\theta}{\sqrt{1 - k^2 sin^2(\theta)}}
    \end{equation}
    and they are called \textbf{Complete if $\phi = \pi/2$}, Incomplete otherwise.
    They can be equivalently represented as:
    \begin{equation}
        F(x; k) = \int^x_0 \frac{dt}{\sqrt{1-k^2 sin^2(\theta)}}
    \end{equation}
    By putting in $ t = sin(\theta) $ and $ x = sin(\phi) $.

    \paragraph{The Solution to the Complete Integrals of the First Kind is a Power Series}
    \begin{eqnarray}
        F(k) &=& \frac{\pi}{2} (\sum_{n=0}^{\infty} (\frac{(2n-1)!!}{(2n)!!}))k^{2n} \\
             &=& \frac{\pi}{2} (\sum_{n=0}^{\infty} (\frac{(2n)!}{2^{2n} (n)!^2})^2 k^{2n}) \\
             &=& \frac{\pi}{2} (1 + (\frac{1}{2})^2 k + (\frac{1 \cdot 3}{2 \cdot 4})^2 k^{2} + ...)
    \end{eqnarray}
\end{definition}



\section{Collisions and Scattering}

In the computations here, it is enough to use Conservation of Energy and Momentum to derive the Mechanics of the system.

\subsection{Disintegration of Particles}

We are interested in the statistical disrtibution of the particles as the system disintegrates. Assuming isotropy over space, we can say from the center of mass frame that the probability of a particle going in the solid angle range $\omega$ to $\omega + d\omega$ is:
\begin{equation}
    d p(n) 
        = \frac{d\omega}{4\pi} 
        = \frac{2\pi sin(\theta_0) d\theta_0}{4\pi} 
        = \frac{1}{2} sin(\theta_0) d\theta_0
\end{equation}



\section{Small Oscillations}

\subsection{Forced Oscillations}

We write the potential expeanded out as a taylor series and see that the first term is a total time derivative and can be ignored from the Lagrangian. The second term is the force.
\begin{eqnarray}
    U = \frac{1}{2} kx^2 + U(x_0, t) + x [\frac{\partial U(x, t)}{\partial t}]_{x=0} \\
    \ddot{x} - \omega^2 x = F(x, t)
\end{eqnarray}



\section{Hamiltons Equations of Motion}


\subsection{The Formulation}

We opt for a different picture from the Lagrangian formulation, seeking to describe the system by first order equitaions of motion, using 2n variable, the position coordinates ($q_i$) and the conjugate momenta ($p_i$) which constitute the phase space.

\begin{equation}
    p_i = \frac{\partial L(q, \dot{q}, t)}{\partial \dot{q}}
\end{equation}



\section{Relativistic Mechanics}


\subsection{Michaelson Morely interferometry}

\subsubsection{The Ether contradiction}

The velocity of light should be same in a ether frame, take earth moving parallel and perpendicular to the said frame.


\subsection{Lorrentz Transforms}

\subsubsection{Derivations}

\paragraph{} Let's try the transform $ \boldsymbol{x^\prime = k(x - vt)} $, since it's linear, one-one, simple, and easily reduces to classical. From the other frame to the first, this is $ \boldsymbol{x = k(x^\prime - vt^\prime)} $. Here we have expolited the first postulate of special relativity.

\paragraph{} Substituting the first equation into the second, we get $ x = k^2(x - vt) + kvt^\prime $
\begin{equation}
    t^\prime = kt + (\frac{1 - k^2}{kv})x
\end{equation}

\paragraph{} Now the calculate k, we shall exploit the second postulate. Replacing the $t^\prime$ by $ x^\prime$ and then $ x $, we get:
\begin{equation*}
    k (x - vt)  = ckt + (\frac{1-k^2}{kv}) cx
\end{equation*}
Solving for x, we get
\begin{equation*}
    ct = x = ct [\frac{1 + \frac{v}{c}}{1 - (\frac{1}{k^2} - 1) \frac{c}{v}}]
\end{equation*}
And from here we get the value of k, and so the Lorrentz Transforms.

\subsubsection{Results}
\begin{eqnarray}
    x^\prime &=& \frac{x - vt}{\sqrt{1 - v^2/c^2}} \\
    t^\prime &=& \frac{t - \frac{vx}{c^2}}{\sqrt{1 - v^2/c^2}} \\
    y^\prime &=& y \\
    z^\prime &=& z
\end{eqnarray}

\subsubsection{Velocity Addition}
We take the derivative of the x-coordinate with respect to time, x and t in the respective frames. We get the following results.
\begin{eqnarray}
    V_x &=& \frac{V_x^\prime + v}{1 + \frac{v V_x^\prime}{c^2}} \\
    V_y &=& \frac{V_y^\prime \sqrt{1-v^2/c^2}}{1 + \frac{v V_x^\prime}{c^2}} \\
    V_z &=& \frac{V_z^\prime \sqrt{1-v^2/c^2}}{1 + \frac{v V_x^\prime}{c^2}}
\end{eqnarray}

\subsubsection{Simulteneity}
Two simulatenous events in one frame are separated by some time in another, which is given by
\begin{eqnarray}
    t_2^\prime - t_1^\prime = \frac{t_0 - v x_2 / c^2}{\sqrt{1 - v^2/c^2}} - \frac{t_0 - v x_1 / c^2}{\sqrt{1 - v^2/c^2}} = \frac{v (x_2 - x_1) / c^2}{\sqrt{1 - v^2/c^2}}
\end{eqnarray}


\subsection{Momentum and Energy}

\subsubsection{Relativistic Momentum}
Particle A be static in frame $S_A$ and Particle B be static in $S_B$. They are spearated by distance Y and thrown towards each other with velocity $V_A$ and $V_B$. Now we have as momenta (Measuring purely in frame $S_A$):
\begin{eqnarray*}
    p_A &=& m_A V_A = m_A (\frac{Y}{T_A}) \\
    p_B &=& m_B V_B = m_B \sqrt{1 - v^2/c^2}(\frac{Y}{T_B})
\end{eqnarray*}
To resolve this, conservation of momentum that is, in both frames, we must have:
\begin{equation}
    m = \frac{m_0}{\sqrt{1 - v^2/c^2}}
\end{equation}
However, since relativistic mass does not make sense, we call the rest mass m and regard this only as an increase in momentum (and not actually in mass).

\subsubsection{Relativistic Energy}

Kinetic Energy (T) = $ \int (v) d(\frac{mv}{\sqrt{1 - v^2/c^2}}) $.
So, we get:
\begin{equation}
    KE = (\gamma - 1) mc^2
\end{equation}
Where we interpret the rest energy as $mc^2$ and the total energy as $E = KE + mc^2 = \gamma mc^2$

\begin{note}{Binomial Approximation}
    The following approxiamtion make calcutation convienient at lower velocities (as compared to light).
    \begin{equation}
        \frac{1}{\sqrt{1 - v^2/c^2}} \approx 1 + \frac{1}{2} \frac{v^2}{c^2} \;\;\;\;\; for \: v << c
    \end{equation}
\end{note}

\subsubsection{Energy and Momentum}
By feeding in the value of $E^2$ and $p^2 c^2$, we can obtain the relation
\begin{equation}
    E^2 = (mc^2)^2 + p^2 c^2
\end{equation}
even if the particle has 0 rest mass.